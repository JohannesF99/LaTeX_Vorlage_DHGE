\documentclass[12pt,a4paper]{scrartcl}
\usepackage[utf8]{inputenc}
\usepackage[german]{babel}
\usepackage{amsmath}
\usepackage{amsfonts}
\usepackage{amssymb}
\usepackage[
  left=3cm,right=2cm,
  top=2.5cm,bottom=2.5cm,
]{geometry}
\usepackage{enumerate}
\usepackage{listings}
\usepackage{tikz}
\usepackage{titlesec}
\usepackage{textcomp}
\usepackage{array}
\usepackage{pgfplots}
\usepackage{booktabs}
\usepackage{adjustbox}
\usepackage{eurosym}
\usepackage{mathtools}
\usepackage{graphicx}
\usepackage{centernot}
\usepackage{verbatim}
\usepackage{fancyhdr}
\usepackage{hyperref}
\usepackage{setspace}
\usepackage{acronym}
\usepackage{ulem}
\usetikzlibrary{automata, positioning, shadows, shapes, snakes, chains, calc, arrows}
\tikzset{initial text={}}
\lstset{frame=single}
\lstset{literate=%
	{Ö}{{\"O}}1
	{Ä}{{\"A}}1
	{Ü}{{\"U}}1
	{ß}{{\ss}}1
	{ü}{{\"u}}1
	{ä}{{\"a}}1
	{ö}{{\"o}}1
}
\DeclareFixedFont{\ttb}{T1}{txtt}{bx}{n}{12}
\DeclareFixedFont{\ttm}{T1}{txtt}{m}{n}{12}
\usepackage{color}
\definecolor{deepblue}{rgb}{0,0,0.5}
\definecolor{deepred}{rgb}{0.6,0,0}
\definecolor{deepgreen}{rgb}{0,0.5,0}
\lstdefinestyle{mystyle}{
    breakatwhitespace=false,         
    breaklines=true,                                                    
    numbers=left,                                    
    showspaces=false,                
    showstringspaces=false,
    showtabs=false,  
    basicstyle=\small,       
}
\newcommand\pythonstyle{\lstset{
language=Python,
basicstyle=\ttm,
otherkeywords={self, sudo},             % Add keywords here
keywordstyle=\ttb\color{deepblue},
emph={__init__},          % Custom highlighting
emphstyle=\ttb\color{deepred},    % Custom highlighting style
stringstyle=\color{deepgreen},
showstringspaces=false
}}
\newcolumntype{L}[1]{>{\raggedright\arraybackslash}p{#1}}
\newcolumntype{C}[1]{>{\centering\arraybackslash}p{#1}}
\newcolumntype{R}[1]{>{\raggedleft\arraybackslash}p{#1}}
\titleformat{\section}{\normalfont\large}{}{\textbf}{\textbf}
\titleformat{\subsection}{\normalfont\large}{}{}{}
\newcommand{\punkte}[1]{\hfill[ \hspace{2mm} / #1] Punkte}
\newcommand{\krel}{\kappa_{\text{rel}}}
\setlength\parindent{0pt}
\DeclarePairedDelimiter{\norm}{\lVert}{\rVert}
\DeclarePairedDelimiter{\abs}{\lvert}{\rvert}
\pagestyle{fancy}
\fancyhead[C]{\textbf{\large{THEMA}}\\ \pagemark}
\fancyhead[LE,RO]{Name, Vorname\\Matrikelnummer\\Duale Hochschule Gera}
\fancyfoot[C]{}
\title{Projektarbeit}
\setcounter{secnumdepth}{0}
\onehalfspacing

%Kopf & Fußzeile
\usepackage{typearea}
\usepackage{fancyhdr}
\renewcommand{\footnotesize}{\fontsize{8pt}{3pt}\selectfont}

\begin{document}

\pythonstyle	
\pagenumbering{Roman}
\setcounter{page}{1}
\phantomsection
\addcontentsline{toc}{section}{Inhaltsverzeichnis}
\tableofcontents 

\newpage
\phantomsection
\addcontentsline{toc}{section}{Abbildungsverzeichnis}
\listoffigures
	
\newpage
\phantomsection
\addcontentsline{toc}{section}{Tabellenverzeichnis}
\listoftables

\newpage
\section{Abkürzungsverzeichnis}
\begin{acronym}[Bash]
\acro{ABKRZ}{Abkürzung}
\end{acronym}

\newpage
\pagenumbering{arabic}
\setcounter{page}{1}
\section{Thema 1}

\subsection{Unterthema 1}
\subsection{Unterthema 2}
Dies ist eine \ac{ABKRZ}.\\
Mathe-Formeln:
\begin{align*}
W &= U \cdot I \cdot t \\
W &= P \cdot t
\end{align*}
\section{Thema 2}
Zahlen im Text gibt man zum Beispiel so an: $500kWh$\\
\\
Eine Tabelle:
\begin{table}[h]
\caption{Jahresverbrauch}
\centering
\begin{tabular}[h]{L{4cm}|c|R{3cm}}
Spalte 1 & Spalte 2 & Spalte 3 \\
\hline
Zeile 1/ Spalte 1 & Zeile 1/ Spalte 2 &Zeile 1/ Spalte 3\\
Zeile 2/ Spalte 1 & Zeile 2/ Spalte 2& Zeile 2/ Spalte 3\\
 \end{tabular}
\end{table}

\section{Thema 3}

\subsection{Unterthema 1}
Eine Aufzählung:\\
\begin{enumerate}
\item[•]Nummer 1
\item[a)]Nummer 2
\end{enumerate}

\newpage
\section{Thema 4}
Verweis auf Bilder etc.: Auf Anlage \ref{Anlage 1} zu finden. \footnote{Fußnote}\\
Code:
\begin{lstlisting}[language=Java, caption=Codeblock 1, style=mystyle, firstnumber=1]
public static void main(){
}
\end{lstlisting}
\subsection{Unterthema 1}
Zitieren geht so. \cite{Test1}\\
Bild einfügen:
\begin{figure}[h]
 \centering
 \includegraphics[width=5cm]{src/bild1.png}
 \caption{Bild 1}
\end{figure}\\

\newpage
	\pagenumbering{Roman}
	\setcounter{page}{5}
	\phantomsection
	\addcontentsline{toc}{section}{Literaturverzeichnis}
	\bibliographystyle{alpha}
	\bibliography{quellen}{}


\newpage
	
	\section{Anlagenverzeichnis}
	\subsection{Anlagen}	

\begin{figure}[h]
 \centering
 \includegraphics[width=13cm]{src/bild1.png}
 \caption{Anlage 1}
 \label{Anlage 1}
\end{figure}

	\newpage
	\phantomsection
	\addcontentsline{toc}{subsection}{Listings}
	\lstlistoflistings

\newpage
	\section{Ehrenwörtliche Erklärung}
\begin{align*}
\textbf{}
\end{align*}
Ich erkläre hiermit ehrenwörtlich,\\
\begin{enumerate}
\item[1.]dass ich meine Projektarbeit mit dem Thema:\\
\\
''\textit{Name}''\\
\\
ohne fremde Hilfe angefertigt habe,
\item[2.]dass ich die Übernahme wörtlicher Zitate aus der Literatur sowie die Verwendung der Gedanken anderer Autoren an den entsprechenden Stellen innerhalb der Arbeit gekennzeichnet habe und
\item[3.]dass ich meine Studienarbeit/Bachelorarbeit bei keiner anderen Prüfung vorgelegt habe.
\end{enumerate}
Ich bin mir bewusst, dass eine falsche Erklärung rechtliche Folgen haben wird.\\
\begin{align*}
\textbf{}
\end{align*}
\begin{tabular}{lp{2em}l}
 \hspace{5cm}   && \hspace{4cm} \\\cline{1-1}\cline{3-3}
 Ort, Datum     && Unterschrift
\end{tabular}
\end{document}